\subsection{Laws of Trigonometry for Triangles}

\subsubsection{Law of Sines}

The Law of Sines relates the ratios of the sides of a triangle to the sines of their opposite angles. It is useful for solving any type of triangle (acute, obtuse, or right). The formula is given by:
\[
\frac{a}{\sin(\alpha)} = \frac{b}{\sin(\beta)} = \frac{c}{\sin(\gamma)}
\]
where:
- \( a \), \( b \), and \( c \) are the lengths of the sides of the triangle,
- \( \alpha \), \( \beta \), and \( \gamma \) are the angles opposite these sides, respectively.

This formula is particularly helpful in the following cases:
- When you know two angles and one side (AAS or ASA case).
- When you know two sides and one non-included angle (SSA case).

\subsubsection{Law of Cosines}

The Law of Cosines relates the lengths of the sides of a triangle to the cosine of one of its angles. This law is useful for solving any type of triangle, especially when you are dealing with non-right triangles. The formula is given by:
\[
c^2 = a^2 + b^2 - 2ab \cdot \cos(\gamma)
\]
where:
- \( a \), \( b \), and \( c \) are the lengths of the sides of the triangle,
- \( \gamma \) is the angle opposite side \( c \).

The Law of Cosines is particularly useful when:
- You know two sides and the included angle (SAS case).
- You know all three sides and want to find an angle (SSS case).
