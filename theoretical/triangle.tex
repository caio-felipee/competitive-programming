\subsection{Triangle Formulas}

\subsubsection{Area of a Triangle}

The area \( A \) of a triangle can be calculated using the base and height:
\[
A = \frac{1}{2} \times \text{base} \times \text{height}
\]
This formula is commonly used when the base and height of the triangle are known.

\subsubsection{Area using Heron’s Formula}

When dealing with a triangle where only the side lengths are known, such as a scalene triangle (where all sides have different lengths), Heron’s formula is useful. It allows the area to be calculated without needing to know the height:
\[
A = \sqrt{s(s-a)(s-b)(s-c)}
\]
where \( s \) is the semi-perimeter, calculated as:
\[
s = \frac{a + b + c}{2}
\]
This formula can be applied to any triangle, whether scalene, isosceles, or equilateral, as long as the side lengths \( a \), \( b \), and \( c \) are known.

\subsubsection{Perimeter of a Triangle}

The perimeter \( P \) of a triangle is simply the sum of the lengths of its sides:
\[
P = a + b + c
\]
This formula is valid for all types of triangles.

\subsubsection{Pythagorean Theorem (Right Triangle)}

In a right-angled triangle, the Pythagorean theorem expresses the relationship between the sides:
\[
c^2 = a^2 + b^2
\]
where \( c \) is the hypotenuse, and \( a \) and \( b \) are the legs. This formula applies only to right triangles.

\subsubsection{Law of Cosines}

The Law of Cosines is particularly useful for finding unknown sides or angles in any triangle, especially when you know two sides and the included angle, or all three sides:
\[
c^2 = a^2 + b^2 - 2ab \cdot \cos(\gamma)
\]
where \( \gamma \) is the angle opposite side \( c \). This is useful for non-right triangles (e.g., scalene or obtuse triangles) where the Law of Sines doesn’t apply directly.

\subsubsection{Law of Sines}

The Law of Sines relates the sides and angles of any triangle:
\[
\frac{a}{\sin(\alpha)} = \frac{b}{\sin(\beta)} = \frac{c}{\sin(\gamma)}
\]
where \( \alpha \), \( \beta \), and \( \gamma \) are the angles opposite sides \( a \), \( b \), and \( c \), respectively. This law is helpful when you know two angles and one side (AAS or ASA case) or two sides and a non-included angle (SSA case).

\subsubsection{Height of a Triangle}

The height \( h \) of a triangle can be determined if the area \( A \) and base are known:
\[
h = \frac{2A}{\text{base}}
\]
This formula is particularly useful when you need to find the height, given the area and base of any triangle.
