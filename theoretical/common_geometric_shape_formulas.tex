\subsection{Common Geometric Shape Formulas}

\subsubsection{Area of a Trapezium}

The area \( A \) of a trapezium (trapezoid) can be calculated using the lengths of the two parallel sides \( a \) and \( b \), and the height \( h \):
\[
A = \frac{1}{2} \times (a + b) \times h
\]

\subsubsection{Area of a Regular Hexagon}

The area \( A \) of a regular hexagon can be calculated using the length of one side \( s \):
\[
A = \frac{3\sqrt{3}}{2} \times s^2
\]
This formula is valid for a regular hexagon, where all six sides are equal.

\subsubsection{Area of a Parallelogram}

The area \( A \) of a parallelogram is calculated by multiplying the base \( b \) by the height \( h \) (the perpendicular distance between the bases):
\[
A = b \times h
\]

\subsubsection{Area of a Rhombus}

The area \( A \) of a rhombus (Losango in portuguese) can be calculated using the lengths of its diagonals \( d_1 \) and \( d_2 \):
\[
A = \frac{1}{2} \times d_1 \times d_2
\]

\subsubsection{Area of an Ellipse}

The area \( A \) of an ellipse is given by the formula:
\[
A = \pi \times a \times b
\]
where \( a \) and \( b \) are the lengths of the semi-major and semi-minor axes, respectively.

\subsubsection{Area of a Regular Pentagon}

The area \( A \) of a regular pentagon with side length \( s \) is given by:
\[
A = \frac{1}{4} \times \sqrt{5(5 + 2\sqrt{5})} \times s^2
\]
This formula is specific to regular pentagons, where all sides and angles are equal.
